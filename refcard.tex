\documentclass{article}

\usepackage[a4paper, landscape, margin=1cm]{geometry}
\usepackage[utf8]{inputenc}
\usepackage[french]{babel}
\usepackage[fontsize=6.5pt]{scrextend}
\usepackage[T1]{fontenc}
%\usepackage[scaled=.7]{beramono} % It looks better without...
\usepackage{hyperref}
\usepackage{multicol}
\usepackage{tabularx}
\usepackage{tabulary}
\usepackage{sectsty}
\usepackage{lmodern}
\usepackage{stix}
\usepackage{parskip}
\usepackage{listings}
\usepackage{xcolor}
\usepackage{multirow}
\usepackage{titlesec}

\input{revision}

% Highlight configuration for C programming language
\lstset{
  language=c,
  breaklines=true,
  keywordstyle=\bfseries\color{black},
  basicstyle=\ttfamily\color{black},
  emphstyle={\em \color{gray}},
  emph={expression, expr, type, NAME, name, expr, value, filename, label, member, type},
  mathescape=true,
  keepspaces=true,
  showspaces=false,
  showtabs=true,
  tabsize=3,
  escapeinside={(*}{*)}
}

% Configuration
\renewcommand{\familydefault}{\sfdefault}

\sectionfont{\fontsize{12}{15}\selectfont}
\subsectionfont{\fontsize{10}{12}\selectfont}

\allsectionsfont{\sffamily\underline}

% No pages numbering
\pagenumbering{gobble}

% Titles and paragraphs more compact
\titlespacing*{\section}{0pt}{1ex plus .2ex minus 0.1ex}{1ex plus .2ex minus 0.1ex}
\titlespacing*{\subsection}{0pt}{1ex plus .2ex minus 0.1ex}{1ex plus .2ex minus 0.1ex}

\newlength\mybaselinestretch
\mybaselinestretch=0pt plus 0.02pt\relax
\addtolength{\baselineskip}{\mybaselinestretch}

\setlength\parindent{0pt}
\setlength\tabcolsep{1.5pt}
\setlength{\columnseprule}{0.4pt}

% Macros
\newcommand{\tab}{\hspace{2em}}
\newcommand{\type}{\emph{type}}
\newcommand{\name}{\emph{name}}
\newcommand{\val}{\emph{value}}
\newcommand{\NAME}{\emph{NAME}}
\newcommand{\file}{\emph{filename}}
\newcommand{\expr}{\emph{expression}}
\newcommand{\etc}{\small \ldots}
\newcommand{\any}{$\hzigzag$~}
\newcommand{\spc}{$\mathvisiblespace$}
\newcommand{\cd}{\lstinline}

\begin{document}

\begin{multicols*}{3}

\begin{center}
  {\Large \bf Carte de référence C99/C11/C18} \\
  HEIG-VD -- version \revision \ -- \revisiondate \\
\end{center}

Cette carte de référence peut être utilisée durant les travaux écrits et examens
des cours \emph{info1} et \emph{info2} à moins que le contraire soit explicitement formulé.
Elle est une liste non exhaustive des possibilités du langage C.

Les prototypes de fonction ont été volontairement simplifiés pour une meilleure lisibilité.

Signification des termes utilisés dans cette carte de référence.

\begin{tabularx}{\linewidth}{
  >{\hsize=0.5\hsize}X% 10% of 4\hsize
  >{\hsize=1.5\hsize}X% 30% of 4\hsize
  >{\hsize=0.5\hsize}X% 30% of 4\hsize
  >{\hsize=1.5\hsize}X% 30% of 4\hsize
     % sum=4.0\hsize for 4 columns
  }

  \tt \etc      & Continuation logique    & \tt \any    & N'importe quoi d'accepté \\
  \tt /\any/    & Expression régulière    & \tt \spc    & Espace obligatoire \\
  \cd{type}     & \tt int, long, float, ... & \cd{name} & \tt /[A-Za-z][A-Za-z0-9\_]+/ \\
  \cd{value}    & Valeur & \cd{NAME} & \tt /[A-Z][A-Z0-9\_]+/ \\
  \cd{filename} & Chemin de fichier relatif & \cd{expression}   & e.g. \tt a + b \\
\end{tabularx}
\hrule
\section*{Généralités}
\begin{tabularx}{\linewidth}{lX}
  Prototype de fonction               & \cd{type name(type name,} \etc \cd{);} \\
  Implantation de fonction         & \cd{type name(type name,} \etc \cd{) \{} \any \cd{\}} \\
  Déclaration de variable            & \cd{type name;} \\
  Fonction principale                & \cd{int main(void) \{} \any \cd{\}} \\
  Fonction principale avec paramètres & \cd{int main(int argc, char* argv[]) \{} \any \cd{\}} \\
  Valeur de retour de la fonction \texttt{main} & \cd{0 == EXIT_SUCCESS}, \cd{1 == EXIT_FAILURE}\\
\end{tabularx}

\section*{Préprocesseur}
\begin{tabularx}{\linewidth}{Xl}
  Inclure bibliothèque globale       & \cd{\#include <filename>} \\
  Inclure bibliothèque locale        & \cd{\#include "filename"} \\
  Remplacer un texte                 & \cd{\#define NAME expression} \\
  Macro                              & \cd{\#define NAME(expr, ...) expression} \\
  ...exemple                         & \cd{\#define max(A,B) ((A)>(B) ? (A) : (B))} \\
  Supprimer une définition           & \cd{\#undef NAME} \\
  Conversion en chaîne de caractères & \cd{\#} \\
  ...exemple                         & \cd{\#define STR(X) \#X} \\
  Concaténation                      & \cd{\#\#} \\
  Exécution conditionelle            & \cd{\#if, \#else, \#elif, \#endif} \\
                                     & \cd{\#ifdef, \#ifndef} \\
  Directive pour le compilateur      & \cd{\#pragma}\tt\spc \any \\
  Génère une erreur de compilation   & \cd{\#error}\tt\spc \any \\
  Est-ce que \texttt{\NAME} est défini ? & \cd{defined(NAME)} \\
  Saut de ligne virtuel              & \cd{\\}
\end{tabularx}

\section*{Types de données}

\subsection*{Modèle de donnée}
\begin{tabularx}{\linewidth}{r|r|r|r|r|l|X}
  \multicolumn{5}{l|}{\bf Profondeur en bits} & \bf Modèle & \bf Système d'exploitation \\
  \cd {short} & \cd{int} & \cd{long} & \cd{long long} & \cd{int*} & & \\
  \hline
  64 & 64 & 64 & 64 & 64 & \bf \tt SILP64 & Classic UNICOS\\
  16 & 32 & 64 & 64 & 64 & \bf \tt LP64   & Unix, Linux, MacOS \\
  16 & 32 & 32 & 64 & 64 & \bf \tt LLP64  & Windows x86-64, MinGW \\
  16 & 32 & 32 & 64 & 32 & \bf \tt ILP32  & Windows x86 \\
  16 & 16 & 32 & 64 & 32 & \bf \tt LP32   & Certains microcontrôleurs 32-bits\\
\end{tabularx}

\subsection*{Type de base}
\begin{tabularx}{\linewidth}{X|l|l|l}
  \bf Description & \bf Type standard & \bf \texttt{<stdint.h>} & \bf Profondeur \\
  \hline
  Caractère                   & \cd{char}      & \cd {int8_t} & 8 bits \\
  \hline
  Entier                      & \cd{int}, \cd{signed}  & \cd {int_least16_t} & $\geq$ 16 bits \\
  Entier court                & \cd{short}     & \cd {int16_t} & $\geq$ 16 bits \\
  Entier long                 & \cd{long}      & \cd {int32_t} & $\geq$ 32 bits \\
  Entier très long            & \cd{long long} & \cd {int64_t} & $\geq$ 64 bits \\
  Entier très long (positif)  & \cd{unsigned long long} & \cd {int64_t} & $\geq$ 64 bits \\
  Entier non signé (positif)  & \cd{unsigned int}, \cd{unsigned} & \cd{uint_least16_t} & $\geq$ 16-bits \\
  \hline
  Simple précision            & \cd{float}     &  & $\geq$ 32 bits \\
  Double précision            & \cd{double}    &  & $\geq$ 32 bits \\
\end{tabularx}

Avec \cd{<stdint.h>}, les types suivants sont disponibles, où \cd{N} est remplacé par un nombre (8, 16, 32, 64, 128), selon disponibilité par l'architecture.

\begin{tabularx}{\linewidth}{Xl}
  Type fixe (p.ex. \texttt{int32\_t} & \texttt{int\textbf{N}\_t}, \texttt{uint\textbf{N}\_t} \\
  Type à taille garantie (p.ex. \texttt{uint\_least8\_t}) & \texttt{int\_least\textbf{N}\_t}, \texttt{uint\_least\textbf{N}\_t} \\
  Type rapide (p.ex. \texttt{int\_fast64\_t}) & \texttt{int\_fast\textbf{N}\_t}, \texttt{uint\_fast\textbf{N}\_t} \\
\end{tabularx}

\begin{tabularx}{\linewidth}{Xl}
  Non défini                 & \cd{void} \\
  Entier non signé pour comptage et indexation & \cd{size_t} \\
  Pointeur sur                       & \cd{*type} \\
  Énumération                        & \cd{enum name \{name=value, ...\}}; \\
  Structure                          & \cd{struct name \{type name; ... \}}; \\
  Union                              & \cd{union name \{type name; ... \}}; \\
\end{tabularx}

\begin{tabularx}{\linewidth}{Xl}
  Type sur structure  & \cd{typedef struct struct_name \{ int a; int b; \} TypeName;} \\
  Structure anonyme   & \cd{struct \{ int c; \} name;} \\
  Structure récursive & \cd{struct Node \{ struct Node* node \} name;} \\
  Liste chaînée       & \cd{typedef struct element \{char *data;} \\
                      & \cd{\ \ struct element *next;\} element, *list, elements[5];}
\end{tabularx}

\section*{Qualificatifs de types}

\begin{tabularx}{\linewidth}{X|l|l}
  \bf Description                     & \bf Mot clé & \bf Exemple \\
  \hline
  Constante                           & \cd{const}    & \cd{const char c = 'u';}\\
  Déclare une variable externe        & \cd{extern}   & \cd{extern void func(void);}\\
  Force une variable dans un registre & \cd{register} & \cd{register int i;}\\
  Réduit la visiblité à ce fichier    & \cd{static}   & \cd{static int array[20];}\\
  Impose aucune optimisation par le compilateur & \cd{volatile} & \cd{volatile int u;}\\
  Limite l'aliasing d'un pointeur     & \cd{restrict} & \cd{restrict int* ptr}\\
  Définition de type                  & \cd{typedef} & \cd{typedef struct\{int a;\} Name;}\\
\end{tabularx}

\section*{Initialisation}
\begin{tabularx}{\linewidth}{Xl}
  Variable & \cd{type name = value;} \\
  Tableau & \cd{type name[] = \{value, ...\}}; \\
  Chaîne de caractère & \cd{type name[] = "...";} \\
  Stucture & \cd{type name = \{.key=value, ...\};} \\
  A zéro & \cd{type name = \{0\};} \\
\end{tabularx}

\section*{Caractères non imprimables et échappements}
\begin{tabularx}{\linewidth}{
  >{\hsize=0.3\hsize}X% 10% of 4\hsize
  >{\hsize=0.2\hsize}X% 10% of 4\hsize
  >{\hsize=1.5\hsize}X% 30% of 4\hsize
  >{\hsize=0.3\hsize}X% 30% of 4\hsize
  >{\hsize=0.2\hsize}X% 30% of 4\hsize
  >{\hsize=1.5\hsize}X% 30% of 4\hsize
     % sum=4.0\hsize for 4 columns
  }

  \cd{\\n}    & \texttt{LF}  & Nouvelle ligne  & \cd{\\r} & \texttt{CR}    & Retour chariot \\
  \cd{\\t}    & \texttt{TAB} & Tabulation horizontale & \cd{\\v}     & \texttt{VT} & Tabulation verticale \\
  \cd{\\f}    & \texttt{FF}  & Nouvelle page   & \cd{\\b} & \texttt{BS} & Retour curseur (\emph{Backspace}) \\
  \cd{\\0}    & \texttt{NUL} & Fin de chaîne de caractère            & \cd{\\a} & \texttt{BEL} & Signal sonore \\
  \cd{\\x...} & & Caractère hexadécimal (\cd{\\x2a})    & \cd{\\0...} & & Caractère octal (\cd{\\030}) \rule{0pt}{3ex} \\
  \cd{\\'}    & & Apostrophe                            & \cd{\\"}    & & Guillemets \\
  \cd{\\?}    & & Point d'interrogation                 & \cd{\\\\}   &  & Backslash \\
\end{tabularx}

\section*{Notation numérique (constantes littérales)}

\begin{tabularx}{\linewidth}{
  >{\hsize=0.5\hsize}X% 10% of 4\hsize
  >{\hsize=1.5\hsize}X% 30% of 4\hsize
  >{\hsize=0.5\hsize}X% 30% of 4\hsize
  >{\hsize=1.5\hsize}X% 30% of 4\hsize
     % sum=4.0\hsize for 4 columns
  }

  \cd{0xff}    & Hexadécimal (\texttt{int}) & \cd{0b110011}    & Binaire (non standard) \\
  \cd{0012}    & Octal & \cd{212}    & Décimale \\
  \cd{12.0f}   & Simple précision & \cd{12.0}    & Double précision \\
  \cd{123u}    & Entier non signé       & \cd{42l}     & \cd{long} \\
  \cd{4312ll}  & \cd{long long} & \cd{12e-3} & Notation en E (0.012) \\ \hline
  \cd{u} ou \cd{U}  & \cd{unsigned int} & \cd{l} ou \cd{L} & \cd{long int} \\
  \cd{ll} ou \cd{LL}  & \cd{long long int} & \cd{ull} & \cd{unsigned long long int} \\
\end{tabularx}

\section*{Pointeurs, Tableaux et Structures}
\begin{tabularx}{\linewidth}{Xl}
  Déclaration d'un pointeur & \cd{type *name;} \\
  Déclaration d'un pointeur de fonction & \cd{type (*name)(type name, ...);} \\
  Pointeur générique & \cd{void* name;} \\
  Adresse nulle & \cd{NULL} \\
  Adresse de variable & \cd{\&name} \\
  Déclaration de tableau & \cd{type name[size];} \\
                         & \cd{type name[size][size];}, ... \\
\end{tabularx}

\subsection*{Structures}
\begin{tabularx}{\linewidth}{Xl}
  Accès membre d'une structure & \cd{name.member} \\
  Accès membre d'une structure (pointeur) & \cd{name->member} \\
  \ \ \cd{(*p).x} et \cd{(p->x)} identique & \\
  Champs de bits & \cd{type member : size} \\
\end{tabularx}

\section*{Opérateurs}

\begin{tabularx}{\linewidth}{c|l|X|c}
  \bf Rang$\dagger$ & \bf Opérateur & \bf Description & \bf Associativité \\
  \hline
  \multirow{6}{*}{1} & \cd{++ --}  & Suffix incrément et décrément $\ddagger$ & \multirow{6}{*}{$\rightarrow$} \\
                     & \cd{()}     & Appel de fonction                     & \\
                     & \cd{[]}     & Indice de tableau                     & \\
                     & \cd{.}      & Accès à un membre de structure        & \\
                     & \cd{->}     & Identique à ci-dessus mais via pointeur & \\
                     & \cd{(type)\{list\}} & \emph{Compound litéral} & \\
  \hline
  \multirow{7}{*}{2} & \cd{++ --}  & Préfixe incrément et décrément $\ddagger$ & \multirow{7}{*}{$\leftarrow$} \\
                     & \cd{+ -}    & Plus, moins unaire (\cd{+2}, \cd{-2}) & \\
                     & \cd{! ~}    & Négation logique et bit à bit & \\
                     & \cd{(type)} & Transtypage (\emph{cast}) & \\
                     & \cd{*}      & Indirection (déréférencement) & \\
                     & \cd{&}      & Adresse de & \\
                     & \cd{sizeof} & Taille de & \\
  \hline
  3                  & \cd{* / \%}  & Multiplication, division et modulo & \multirow{11}{*}{$\rightarrow$} \\
  \cline{1-3}
  4                  & \cd{+ -}    & Addition, soustraction & \\
  \cline{1-3}
  5                  & \cd{<< >>}  & Décalage à gauche, à droite & \\
  \cline{1-3}
  \multirow{2}{*}{6}  & \cd{< <=}   & Plus petit, plus petit ou égal & \\
                     & \cd{> >=}   & Plus grand, plus grand ou égal & \\
  \cline{1-3}
  7                  & \cd{== !=}  & Égal, différent de & \\
  \cline{1-3}
  8                  & \cd{&}      & ET bit à bit & \\
  \cline{1-3}
  9                  & \cd{^}      & XOR bit à bit (ou exclusif) & \\
  \cline{1-3}
  10                 & \cd{|}      & OU bit à bit & \\
  \cline{1-3}
  11                 & \cd{&&}     & ET logique & \\
  \cline{1-3}
  12                 & \cd{||}     & OU logique & \\
  \hline
  13                 & \cd{?:}     & Opérateur ternaire & \multirow{6}{*}{$\leftarrow$} \\
  \cline{1-3}
  \multirow{5}{*}{14} & \cd{=}      & Assignement simple & \\
                     & \cd{+= -=}  & Assignement avec somme, différence & \\
                     & \cd{*= /= \%=} & Assignement avec multiplication, division, modulo & \\
                     & \cd{<<= >>=}  & Assignement avec décalage & \\
                     & \cd{&= ^= |=} & Assignement avec opération bit à bit & \\
  \hline
  15                 & \cd{,}     & Virgule & $\rightarrow$\\
\end{tabularx}

$\dagger$ Plus le rang est petit, plus l'opérateur est prioritaire. \\
$\ddagger$ Opérateur est pris en compte selon priorité, mais action décalée au point de séquence suivant/précédent.

\section*{Flot de contrôle}

\begin{tabularx}{\linewidth}{
  >{\hsize=1.3\hsize}X%
  >{\hsize=0.7\hsize}X%
  >{\hsize=1.3\hsize}X%
  >{\hsize=0.7\hsize}X%
  }

  \multicolumn{3}{l}{Fin d'instruction (instruction vide)} & \cd{;} \\
  \multicolumn{3}{l}{Délimiteur de blocs} & \cd{\{ \}} \\
  \multicolumn{3}{l}{Interruption d'un \cd{switch}, \cd{do}, \cd{for}} & \cd{break} \\
  \multicolumn{3}{l}{Continue à la prochaine itération d'un \cd{while}, \cd{do}, \cd{for}} & \cd{continue} \\
  Saut à une étiquette & \cd{goto label} & Étiquette & \cd[emph={name}]{name:} \\
  Retour de fonction & \cd{return expr;} & & \\
  \end{tabularx}

  \begin{tabularx}{\linewidth}{Xl}
  Si, sinon si, sinon & \cd{if (expr) \{}~\any\cd{\} else if (expr) \{}~\any\cd{\} else \{}~\any\cd{\};} \\
  Tant que & \cd{while (expr) \{}\any\cd{\}} \\
  Pour & \cd{for (expr, expr, expr) \{}\any\cd{\}} \\
  Jusqu'à ce que & \cd{do \{}\any\cd{\} while (expr);} \\
  Choix multiples & \cd{switch(expr) \{ case a; instr; break; case b: instr; \}} \\
  \end{tabularx}

\section*{Bibliothèques Standards}
  \begin{tabularx}{\linewidth}{XXXXX}
    \cd{<assert.h>} & \cd{<complex.h>}  & \cd{<ctype.h>}  & \cd{<errno.h>}   & \cd{<fenv.h>} \\
    \cd{<float.h>}  & \cd{<inttypes.h>} & \cd{<limits.h>} & \cd{<locale.h>}  & \cd{<math.h>} \\
    \cd{<setjmp.h>} & \cd{<signal.h>}   & \cd{<stdarg.h>} & \cd{<stdbool.h>} & \cd{<stddef.h>} \\
    \cd{<stdint.h>} & \cd{<stdio.h>}    & \cd{<stdlib.h>} & \cd{<string.h>}  & \cd{<tgmath.h>} \\
    \cd{<time.h>}   &              &            &             & \\
  \end{tabularx}

\section*{Mots clés du langage}
  \begin{tabularx}{\linewidth}{XXXXX}
    \cd{auto} & \cd{double}  & \cd{int}  & \cd{struct}   & \cd{break} \\
    \cd{else}  & \cd{long} & \cd{switch} & \cd{case}  & \cd{enum} \\
    \cd{register} & \cd{typedef}   & \cd{char} & \cd{extern} & \cd{return} \\
    \cd{union} & \cd{const}    & \cd{float} & \cd{short}  & \cd{unsigned} \\
    \cd{continue}   & \cd{for} & \cd{static} & \cd{void} & \cd{default} \\
    \cd{goto} & \cd{sizeof} & \cd{volatile} & \cd{do} & \cd{if} \\
    \cd{signed} & \cd{while} & \cd{_Bool} & \cd{_Imaginary} & \cd{_Complex} \\
    \cd{restrict} & \cd{inline} &  &  &  \\
  \end{tabularx}
\section*{Tests de classe de caractères \texttt{<ctype.h>}}

\begin{tabularx}{\linewidth}{X|r|l}
  \bf Description         & Condition              & Prototype \\
  \hline
  Alphanumérique ?        & \texttt{/[0-9a-zA-Z]/} & \cd{int isalnum(int c)} \\
  Alphabétique ?          & \texttt{/[a-zA-Z]/} & \cd{int isalpha(int c)} \\
  Chiffre                 & \texttt{/[0-9]/} & \cd{int isdigit(int c)} \\
  Caractère de contrôle ? & \footnotesize $[0\dots32, 127]$  & \cd{int iscntrl(int c)} \\
  Caractère imprimable    & \footnotesize $[33\dots 126]$ & \cd{int isgraph(int c)} \\
  Caractère imprimable avec espace & \footnotesize$[32\dots 126]$ & \cd{int isprint(int c)} \\
  Minuscule ?             & \footnotesize $[65\dots 90]$ & \cd{int islower(int c)} \\
  Majuscule ?             & \footnotesize $[97\dots 122]$ & \cd{int isupper(int c)} \\
  Espaces ?               & \footnotesize $[9\dots13,32]$ & \cd{int isspace(int c)} \\
  Hexadécimal ?           & \texttt{/0-9a-fA-F/} &  \cd{int isxdigit(int c)} \\
  \multicolumn{2}{l|}{Ponctuation ? \hfill\footnotesize $[33\dots 47, 58\dots64, 91\dots96, 123\dots126]$} & \cd{int ispunct(int c)} \\
  \hline
  Conversion minuscule à majuscule & \texttt{/a-z/} $\to$ \texttt{/A-Z/} & \cd{int toupper(int c)} \\
  Conversion majuscule à minuscule & \texttt{/A-Z/} $\to$ \texttt{/a-z/} & \cd{int tolower(int c)} \\
\end{tabularx}

\section*{Opération sur les Chaînes de caractères \texttt{<string.h>}}

\begin{tabularx}{\linewidth}{Xl}
  Longueur de \texttt{s} & \cd{size_t strlen(char *s)} \\
  Copie \texttt{src} dans \texttt{dest} & \cd{char *strcpy(char* dest, char* src)} \\
   ... jusqu'à \texttt{n} caractères & \cd{char *strncpy(char* dest, char* src, size_t n)} \\
  Concatène \texttt{src} après \texttt{dest} & \cd{char *strcat(char* dest, char* src)} \\
   ... jusqu'à \texttt{n} caractères & \cd{char *strncat(char* dest, char* src, size_t n)} \\
  Compare \texttt{s1} à \texttt{s2} & \cd{int strcmp(char *s1, char *s2)} \\
   ... seulement \texttt{n} caractères & \cd{int strncmp(char *s1, char *s2, size_t n)} \\
  Pointeur sur le premier \texttt{c} dans \texttt{s} & \cd{char *strchr(char *s, int c)} \\
  Pointeur sur le dernier \texttt{c} dans \texttt{s} & \cd{char *strrchr(char *s, int c)} \\
  \hline
  Copie \texttt{n} char de \texttt{s2} dans \texttt{s1} & \cd{void *memcpy(void* s1, void* s2, size_t n)} \\
  Copie \texttt{n} char de \texttt{s2} dans \texttt{s1} $\dagger$ & \cd{void *memmove(void* s1, void* s2, size_t n)} \\
  Compare \texttt{n} char de \texttt{s2} avec \texttt{s1} & \cd{int memcmp(void *s1, void *s2, size_t n)} \\
  Cherche \texttt{c} dans \texttt{s} sur \texttt{n} char & \cd{void *memchr(void *s, int c, size_t n)} \\
  Initialise \texttt{s} de longueur \texttt{n} à \texttt{c} & \cd{void *memset(void *s, int c, size_t n)} \\
\end{tabularx}

$\dagger$ Permet le chauvauchement des domaines mémoire de source et de destination.

\section*{Entrée/Sortie \texttt{<stdio.h>}}

\begin{tabularx}{\linewidth}{
  >{\hsize=0.5\hsize}X% 10% of 4\hsize
  >{\hsize=1.5\hsize}X% 30% of 4\hsize
  >{\hsize=0.5\hsize}X% 30% of 4\hsize
  >{\hsize=1.5\hsize}X% 30% of 4\hsize
  }

  \cd{FILE* fp} & Pointeur de fichier & \cd{stdin} & Entrée standard \\
  \cd{stdout}   & Sortie standard     & \cd{stderr} & Sortie erreur \\
  \cd{EOF}      & Fin de fichier      & \cd{BUFSIZ} & Buffer size \\
\end{tabularx}

\begin{tabularx}{\linewidth}{Xl}
  Lire un caractère & \cd{int getchar(void)} \rule{0pt}{3ex} \\
  Écrire un caractère & \cd{int putchar(int c)} \\
  Écriture formatée & \cd{int printf(char* format, ...)} \\
  Lecture formatée & \cd{int scanf(char* format, ...)} \\
  Lecture dans la chaîne \texttt{s} & \cd{int sscanf(char* s, char* format, ...)} \\
  Affiche la chaîne \texttt{s} & \cd{int puts(char *s)} \\
  Écrire dans la chaîne \texttt{s} & \cd{int sprintf(char* s, char* format, ...)}  \\
  Ouvre un fichier & \cd{FILE *fopen(char* filename, char* mode)} \\
  Ferme un fichier & \cd{int fclose(FILE *fp)} \\
  Lire un caractère & \cd{int fgetc(FILE *fp)} \\
  Écrire un caractère & \cd{int fputc(int c, FILE *fp)} \\

  Position du curseur & \cd{long int ftell(FILE *fp)} \\
  Déplace le curseur & \cd{int fseek(FILE *fp, long int offset, int whence)} \\

  Lire une ligne & \cd{char *fgets(char* s, int n, FILE* fp)} \\
  Écrire une ligne & \cd{int fputs(char* s, FILE* fp)} \\

  Écriture avec gabarit & \cd{int fprintf(FILE* fp, char* format, ...)} \\
  Lecture avec gabarit & \cd{int fscanf(FILE* fp, char* format, ...)} \\

  Retourne \cd{0} si pas d'erreur & \cd{int ferror(FILE *fp)} \\
  Retourne \cd{!0} si fin de fichier & \cd{int feof(FILE *fp)} \\
\end{tabularx}

\begin{tabularx}{\linewidth}{Xl}
  Lecture & \cd{size_t fread(void *p, size_t size, size_t n, FILE *fp)} \\
  Écriture & \cd{size_t size_t fwrite(void *p, size_t size, size_t n, FILE *fp)} \\
\end{tabularx}

\begin{tabularx}{\linewidth}{
  >{\hsize=0.2\hsize}X >{\hsize=1.8\hsize}X
  >{\hsize=0.2\hsize}X >{\hsize=1.8\hsize}X
  >{\hsize=0.2\hsize}X >{\hsize=1.8\hsize}X
  }
  \multicolumn{4}{l}{\bf Modes de lecture (\texttt{mode})} \\
  \tt r & Lecture & \tt w & Écriture (écrase éxistant) & \tt a & Ajout \\
  \tt rb & Lecture Binaire & \tt wb & Écriture Binaire & \tt ab & Ajout Binaire \\
  \tt r+ & Lecture, écriture & \tt w+ & Écriture + efface fichier & \tt a+ & Ouvre ou crée fichier \\
  \tt rb+ & Identique mais binaire & \tt wb+ & Identique mais binaire & \tt ab+ & Identique mais binaire \\
\end{tabularx}

\subsection*{Formattage de chaîne de caractères}

Format pour \texttt{printf}: \texttt{\%[flags][width|*][.precision|*][length]type} \\
Format pour \texttt{scanf}: \texttt{\%[*][width][length]type} \\

\texttt{*}: Taille passée en argument (printf), Supprime la capture (scanf)

\begin{tabularx}{\linewidth}{
  >{\hsize=0.2\hsize}X>{\hsize=1.8\hsize}X
  >{\hsize=0.2\hsize}X>{\hsize=1.8\hsize}X
  }
  \multicolumn{4}{l}{\bf Drapeaux (\texttt{flags})} \\
  \cd{-}        & Justifie à gauche & \cd{+}    & Affiche le signe \texttt{+} \\
  \tt \spc & Remplissage avec des espace (défaut)       & \cd{0}      & Remplissage avec des zéros \\
\end{tabularx}

\begin{tabularx}{\linewidth}{
  >{\hsize=0.2\hsize}X>{\hsize=1.8\hsize}X
  >{\hsize=0.2\hsize}X>{\hsize=1.8\hsize}X
  }
  \multicolumn{4}{l}{\bf Longueur (\texttt{length})} \\
  \cd{hh} & Entier promu d'un \cd{char} & \cd{h}    & Entier promu d'un \cd{short} \\
  \cd{ll} & Entier \cd{long long}       & \cd{l}    & Entier \cd{long} \\
  \cd{L} & Flottant double précision    & \cd{z}    & Entier \cd{size_t} \\
  \cd{j} & Entier \cd{intmax_t}         & \cd{t}    & Entier \cd{ptrdiff_t} \\
\end{tabularx}

\begin{tabularx}{\linewidth}{
  >{\hsize=0.2\hsize}X% 10% of 4\hsize
  >{\hsize=1.8\hsize}X% 30% of 4\hsize
  >{\hsize=0.2\hsize}X% 30% of 4\hsize
  >{\hsize=1.8\hsize}X% 30% of 4\hsize
     % sum=4.0\hsize for 4 columns
  }
  \multicolumn{4}{l}{\bf Champ type (\texttt{type})} \\
  \cd{\%}        & Caractère pourcent & \cd{c}    & Caractère (\texttt{char}) \\
  \cd{d}, \cd{i} & Entier signé       & \cd{u}      & Entier non signé \\
  \cd{o} & Entier non signé en octal  & \cd{x}, \cd{X} & Entier non signé en hexadécimal \\
  \cd{f}, \cd{F} & \cd{float} et \cd{double} (\cd{\%lf} avec \cd{scanf}) & \cd{a}, \cd{A}    & Double en hexadécimal \\
  \cd{p}         & Pointeur (adresse) & \cd{n}    & Nombre de caractère écrits \\
  \cd{e}, \cd{E} & \multicolumn{3}{l}{Double en format exponentiel \texttt{[-]d.ddd e [+-]ddd}}\\
  \cd{g}, \cd{G} & \multicolumn{3}{l}{Double en format le plus adapté normal (\cd{f}, \cd{F}) ou exponentiel (\cd{e}, \cd{E})} \\
\end{tabularx}

\section*{Utilitaires Standard \texttt{<stdlib.h>}}

\begin{tabularx}{\linewidth}{Xl}
  \multicolumn{2}{l}{\bf Macros} \\
  Code de fin: succès & \cd{EXIT_SUCCESS} \\
  Code de fin: erreur & \cd{EXIT_FAILURE} \\
  Pointeur null & \cd{NULL} \\
  Valeur aléatoire maximum & \cd{RAND_MAX} \\
\end{tabularx}

\begin{tabularx}{\linewidth}{
  >{\hsize=0.2\hsize}X% 10% of 4\hsize
  >{\hsize=1.8\hsize}X% 30% of 4\hsize
  >{\hsize=0.2\hsize}X% 30% of 4\hsize
  >{\hsize=1.8\hsize}X% 30% of 4\hsize
     % sum=4.0\hsize for 4 columns
  }
  \multicolumn{4}{l}{\bf Suffix pour fonctions \cd{ato}\texttt{\spc} et \cd{strto}\texttt{\spc}}  \\
  \cd{f}  & \cd{float} & \cd{d} & \cd{double} \\
  \cd{i}  & \cd{int} & \cd{l} & \cd{long} \\
  \cd{ll}  & \cd{long long} & \cd{ld} & \cd{long double} \\
  \cd{ul}  & \cd{unsigned long} & \cd{ull} & \cd{unsigned long long} \\
\end{tabularx}

\begin{tabularx}{\linewidth}{Xl}
  \multicolumn{2}{l}{\bf Conversion de chaînes de caractères en nombre} \\
  \cd{f}, \cd{i}, \cd{l}, \cd{ll} & \cd{type ato}\texttt{\spc}\cd{ (char* str);} \\
  \cd{f}, \cd{l}, \cd{d}, \cd{ld}, \cd{ll}, \cd{ul}, \cd{ull} & \cd{type strto}\texttt{\spc}\cd{ (char* str, char** endptr);} \\

  \multicolumn{2}{l}{\bf Arithmétique d'entiers} \rule{0pt}{3ex}\\
  Valeur absolue & \cd{int abs (int n);} \\
  Valeur absolue long & \cd{long labs (long n);} \\
  Valeur absolue long long & \cd{long long llabs (long long n);} \\
  Division entière & \cd{int div (int n);} \\
  Division entière long & \cd{long ldiv (long n);} \\
  Division entière long long & \cd{long long lldiv (long long n);} \\

  \multicolumn{2}{l}{\bf Pseudo aléatoire} \rule{0pt}{3ex}\\
  Nombre aléatoire \cd{[0,MAX_RAND]} & \cd{int rand (void);} \\
  Initialise le générateur aléatoire & \cd{void srand (unsigned int seed);} \\

  \multicolumn{2}{l}{\bf Environement} \rule{0pt}{3ex}\\
  Termine le processus & \cd{void exit (int status);} \\
  Termine le processus avec \cd{SIGABRT} & \cd{void abort(void);} \\
  Exécute une commande système & \cd{int system (const char* command);} \\
  Récupère l'environment & \cd{char* getenv (const char* name);} \\

  \multicolumn{2}{l}{\bf Allocation de mémoire dynamique} \rule{0pt}{3ex} \\
  Alloue un tabeau initialisé & \cd{void* calloc (size_t num, size_t size);} \\
  Alloue un bloc mémoire & \cd{void* malloc (size_t size);} \\
  Réalloue un bloc mémoire & \cd{void* realloc (void* ptr, size_t size);} \\
  Désalloue un bloc mémoire & \cd{void free (void* ptr);} \\
\end{tabularx}
\begin{tabularx}{\linewidth}{Xl}
  \multicolumn{2}{l}{\bf Tri et recherche} \rule{0pt}{3ex} \\
  Recherche dichotomique & \cd{void* bsearch (void* key, void* base, size_t num,} \\
                         & \cd{\ \ \ \ size_t size, int (*compar)(void*, void*));} \\
  Tri rapide             & \cd{void qsort (void* base, size_t num,} \\
                         & \cd{\ \ \ \ size_t size, int (*compar)(void*, void*));} \\
\end{tabularx}

\section*{Date et heure \texttt{<time.h>}}
\begin{tabularx}{\linewidth}{Xl}
  Temps absolu en seconde depuis 1970/01/01 & \cd{time_t} \\
  Temps relatif depuis un point dans le temps (\emph{boot}) & \cd{clock_t} \\
  Temps calendrier (date et heure) & \cd{struct tm} \\
\end{tabularx}

\begin{tabularx}{\linewidth}{Xl}
  Temps écoulé depuis le début & \cd{clock_t clock(void)} \\
  Temps calendrier courante & \cd{time_t time(time_t *timer)} \\
  Différence de temps en secondes & \cd{double difftime(time_t time1, time_t time2)} \\
  Conversion en temps calendrier & \cd{time_t mktime(struct tm *timeptr)} \\
  Conversion d'une temps calendrier & \cd{struct tm *gmtime(const time_t *timer)} \\
  Conversion d'une temps locale & \cd{struct tm *localtime(const time_t *timer)} \\
  Représentation texte & \cd{char *asctime(const struct tm *timeptr)} \\
  Représentation texte & \cd{char *ctime(const time_t *timer)} \\
  Gabarit d'un temps & \cd{size_t strftime(char *str, size_t maxsize,} \\
                     & \cd{\ \ \ \ char *format, struct tm *timeptr)} \\
\end{tabularx}

{
\small
\begin{lstlisting}
struct tm {
  int tm_sec;   // Secondes (0..59)
  int tm_min;   // Minutes (0..59)
  int tm_hour;  // Heures (0..23)
  int tm_mday;  // Jour du mois (1..31)
  int tm_mon;   // Mois (0..11)
  int tm_year;  // Nombre d'années depuis 1900
  int tm_wday;  // Jour depuis dimanche (0..6)
  int tm_yday;  // Jour dans l'année de (0..365)
  int tm_isdst; // Heure été ? (Daylight Saving Time)
};
\end{lstlisting}
}

Format pour \texttt{asctime}: \cd{"\%.3s \%.3s\%3d \%.2d:\%.2d:\%.2d \%d\\n"}

\section*{Fonctions Mathématique \texttt{<math.h>}}

Si non spécifé, le prototype est la forme \cd{double func(double x)}.

\begin{tabularx}{\linewidth}{Xl}
  Fonctions trigonométriques & \cd{cos}, \cd{sin}, \cd{tan} \\
  Fonctions inverses (\emph{arc}) & \cd{acos}, \cd{asin}, \cd{atan} \\
  Fonctions hyperbliques & \cd{cosh}, \cd{sinh}, \cd{tanh} \\
  Arctangente corrigée $\arctan(y / x)$ & \cd{double atan2(double y, double x)} \\
  Fonctions exponentielles & \cd{exp}, \cd{log}, \cd{log10} \\
  Division (retourne partie décimale) & \cd{double modf(double x, double *integer)} \\
  Reste de division $x/y$ pour $x - n \cdot y$ & \cd{double fmod(double x, double y)} \\
  Arrondi à l'entier inférieur & \cd{floor} \\
  Arrondi à l'entier supérieur & \cd{ceil} \\
  Valeur absolue & \cd{fabs} \\
  Puissance & \cd{double pow(double x, double y)} \\
  Racine carrée & \cd{sqrt} \\
\end{tabularx}

\section*{Booléens \texttt{<stdbool.h>}}

\cd{bool}, \cd{true} (vrai (1)), \cd{false} (faux (0))

\section*{Limites \texttt{<limites.h>} / \texttt{<float.h>}}

Valeurs typiques pour un modèle de donnée \textbf{LP32}.

\begin{tabularx}{\linewidth}{
  >{\hsize=0.9\hsize}X% 10% of 4\hsize
  >{\hsize=1.2\hsize}X% 30% of 4\hsize
  >{\hsize=0.9\hsize}X% 30% of 4\hsize
  >{\hsize=1.2\hsize}X% 30% of 4\hsize
     % sum=4.0\hsize for 4 columns
  }
  \cd{CHAR_BIT}  & (8) &   & \\
  \cd{SCHAR_MAX} & +127 & \cd{SCHAR_MIN} & -128 \\
  \cd{CHAR_MAX} & \cd{SCHAR_MAX} & \cd{CHAR_MIN} & \cd{SCHAR_MIN} \\
  \cd{SHRT_MAX} & +32'767 & \cd{SHRT_MIN} & -32'768 \\
  \cd{INT_MAX} & +2'147'483'647 & \cd{INT_MIN} & -2'147'483'648 \\
  \cd{LONG_MAX} & +2'147'483'647 & \cd{LONG_MIN} & -2'147'483'648 \\
  \cd{LLONG_MAX} & +2e63 - 1 & \cd{LLONG_MIN} & -2e63 \\
  \cd{UCHAR_MAX} & +255 & \cd{UINT_MAX} & +4'294'967'295 \\
  \cd{ULONG_MAX} & +4'294'967'295 & \cd{ULLONG_MAX} & +2e64 - 1 \\
  \cd{FLT_MAX} & \cd{3.4e38} & \cd{FLT_MIN} & \cd{1.2e-38} \\
  \cd{DBL_MAX} & \cd{1.8e308} & \cd{DBL_MIN} & \cd{2.2e-308} \\
  \cd{FLT_EPSILON} & \cd{1.1e-7} & \cd{DBL_EPSILON} & \cd{2.2e-16}
\end{tabularx}

% \section*{Compilation (gcc/clang)}
% \begin{tabularx}{\linewidth}{Xl}
%   \texttt{-O2 -std=c99 -Wall -Wextra -Wconversion} & Options standards \\
%   \texttt{-g} & Débug \\
%   \texttt{cc filename.c} & Compilation simple \\
%   \texttt{cc -o name filename.c} & Compilation simple, \\
%   & choix du nom de l'exécutable \\
% \end{tabularx}

\end{multicols*}
\end{document}
